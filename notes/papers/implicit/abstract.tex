
\begin{abstract}
In a type theory based proof checker, meta variables have a number of important
applications, such as implicit syntax, interaction, and proof search.
Unfortunately type checking in the presence of meta variables is not without
complications.  In a dependently typed setting, type checking needs to evaluate
arbitrary terms and so to ensure normalisation an important invariant is that
the terms that are evaluated are type correct. Meta variables risk breaking this
invariant, since the type correctness of a term might depend on the
instantiation of a particular meta variable.

We present a type checking algorithm for a dependently typed logical framework
with meta variables and prove that it is sound and decidable. To ensure type
correctness of evaluated terms the type checker works with well-typed
approximations of terms and only refine an approximated term when this is known
to be type correct.
\end{abstract}

