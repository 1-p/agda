\documentclass{article}

\usepackage{hcar}

\begin{document}

% Agda-NA.tex
\begin{hcarentry}[section,updated]{Agda}
\label{agda}
\report{Ulf Norell}%11/13
\status{actively developed}
\participants{Ulf Norell, Nils Anders Danielsson, Andreas Abel,
Jesper Cockx, Makoto Takeyama,
Stevan Andjelkovic, Jean-Philippe Bernardy, James Chapman,
Dominique Devriese, Peter Divianszki,
Fredrik Nordvall Forsberg, Olle Fredriksson, Daniel Gustafsson,
Alan Jeffrey, Fredrik Lindblad, Guilhem Moulin, Nicolas Pouillard, Andrés Sicard-Ramírez
and many others}
\makeheader

Agda is a dependently typed functional programming language (developed
using Haskell). A central feature of Agda is inductive families,
i.e., GADTs which can be indexed by \emph{values} and not just types.
The language also supports coinductive types, parameterized modules,
and mixfix operators, and comes with an \emph{interactive}
interface---the type checker can assist you in the development of your
code.

A lot of work remains in order for Agda to become a full-fledged
programming language (good libraries, mature compilers, documentation,
etc.), but already in its current state it can provide lots of value as a
platform for research and experiments in dependently typed programming.

Some highlights from the past six months:
\begin{itemize}
\item Agda~2.5.2 was released in December 2016.
\item The Agda documentation at
\url{http://agda.readthedocs.org/en/stable/} is being continuously improved.
\item Experimental support for homotopy type theory has been added to the
developement branch by Andrea Vezzosi.
\end{itemize}
Release of Agda~2.5.3 is planned for summer 2017.

\FurtherReading
  The Agda Wiki: \url{http://wiki.portal.chalmers.se/agda/}
\end{hcarentry}

\end{document}
